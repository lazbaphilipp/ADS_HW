\chapter{Создание топологии} \label{chap:altium-PcbDoc}


\section{Создание и настройка правил}

Применённые стандартные правила приведены в следующем списке:
\begin{itemize}
	\item Зазор --- 0.15 мм
	\item Защита от КЗ
	\item Запрет на свободные и висящие концы и пины.
	\item Ширина
		\begin{itemize}
			\item ВЧ линий: 0.2-1.1-1.15~мм
			\item Остальных линий: 0.2-0.3-0.5~мм
		\end{itemize}
	\item ВЧ углы скругляются с двух сторон, остальные срезаются.
	\item Переходные отверстия 0.8х0.4~мм
	\item Зазор  от металлизации  до  выреза паяльной маски 0.05~мм
	\item Расстояние от полигона до элемента 0.2~мм
	\item Угол минимум  60°  для  всех видов металлизированных примитивов. 
	\item Отверстия от 0.4 до 3.5~мм с минимальным шагом в 0.2~мм
	\item Минимальный мостик паяльной маски --- 0.125~мм
	\item Зазор 0.15~мм для шелкографии
	\item Установим расстояние в 0.25~мм от края платы до любого элемента, кроме полигонов --- для торцевой металлизации зазор между платой и полигоном должен быть равен 0.
	\item Минимальные зазоры для 3D моделей --- 0.2~мм по Ox и 0.1~мм по Oy
\end{itemize}

Конфигурация слоёв платы остаётся такой же как и у посадочных мест.

Цветовая схема - "Неважно какого цвета ваша плата, главное чтобы синяя!".

\section{Разводка печатной платы}


\section{Итоговый вид}


